\documentclass{article}

\title{Chapters 4 to 7}
\author{Vladimir Feinberg}

\input{../defs}

\begin{document}

\maketitle

Notes on Chapters 4 to 7 of Pearl and Mackenzie's \textit{Book of Why} \cite{pearl2018book}.

\setcounter{section}{3}

\section{Confounding}

The basic confounder (Fig.~\ref{fig:fig41}).

\begin{figure}[h]
  \centering
  \includegraphics{} %% TODO
  \caption{\label{fig:fig41} }
\end{figure}


(Working through the examples in book of why ch 4), just go through them all.

What to do when there are multiple paths from X to Y with no colliders? How to compute do(X) ---> only remove all in-arrows.
* would be double-counting causal effects
* E.g., add arrow A -> D to game 2

Backdoor is all paths that have in-arrows into X.

Use rules a-d to de-confound.

game 5 instead of having C, what if you had an arrow Y->B (cyclic). or B->Y. then not solvable.

* Do cycles appear in practice, or can you "unwind them" (successfull enterpreneur -> funding -> exit)

-----

Worked smoking example

So do you put an arrow diagram 5.2 -> 5.1 or not? Need sensitivity analysis (Cornfield) -- how strong would gene smoking need to be in order to explain.

how small does the value found in the arrow have to be before we should ignore it as non-existent?

----

Paradoxes

monty hall bias

simpson's paradox

obserational non rct bias --> collider

"if you know you're a man, you shouldn't take the drug, ifyou know you're a woman, you shouldn't, etc."

but people in general should take drugs.

The problem is we think of every association causally (regardless of the summarization dimension). But different summarizations can conflict in causal prescription. The root problem is you can't go to a causal prescription (take the drug in order to get better).

Actually answering that question requires a causal model. (tbl 6.4)

Lord's paradox = simpsons but numerical. Could be fixed

can we measure the error of the diagram? critical for making trial observation choices.

-----

work an example of backdoor adjustment for a binary confounder.

frontdoor adjustment work out ex  in 7.1 (for removing mediator effect)



do calculus

3 rules

they are complete, everything that a causal network implies, they'll derive.

What is known
* can reduce all reducible  do expressions into observational ones (decidable; not every one can be removed for a given study)
* can extend this to enabling randomization of possible subgroups of other variables
